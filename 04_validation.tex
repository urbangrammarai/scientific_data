Spatial signatures are unique as a classification method, limiting the potential
validation methods to only indirect methods using ancillary datasets capturing
conceptually similar aspects of environment. We compare the signatures with three of
such datasets, each focusing on a different classification perspective, but all related
to our classification to a degree when we can assume there will be measurable level of
association between the two:

\begin{itemize}
    \item WorldPop settlement patterns of building footprints (2021)
    \item Classification of Multidimensional Open Data of Urban Morphology (MODUM)(2015)
    \item Copernicus Urban Atlas (2018)
\end{itemize}


\subsection*{Validation method}
% General method of validation
    % data transfer (one or the other way depending on feasibility)
    % chi-squared statistic
    % Cramer's V
All datasets, spatial signatures as well as those selected as validation contain
categorical classification of space linked to their unique geometry. The first task, to
make each pair comparable is to transfer data to the same geometry. That can be
interpolation of one set of polygon-based data to another (input to ETCs) or converting
spatial signatures to the raster representation matching an input raster as the latter
is computationally more feasible when one of the layers is already a raster. The second
step is a statistical comparison of two sets of classification labels, one representing
spatial signature typology and the other validation classes. We use contingency tables
and a Pearson's $\chi^{2}$ test to determine whether the frequencies of observed
(signature types) and expected (validation types) labels significantly differ in one or
more categories. Furthermore, we use Cramér's $V$ statistics to assess the strength of
an association (assuming the Pearson's $\chi^{2}$ test rejected the hypothesis of
independence).

\subsection*{WorldPop settlement patterns of building footprints}
% - WorldPop (Spsig)
    % description of dataset + figure
WorldPop settlement patterns of building footprints aim to derive a typology of
morphological patterns based on the gridded approach (spatial unit is a grid of a size
100x100m per cell) and building footprints. Authors measure 6 morphometric characters
linked to the grid cells and use them as an input of unsupervised clustering leading to
a 6 class typology (Figure XXX).
    % expectations regarding similarity
As the classification is dependent on the building footprint data, grid cells that do
not contain any information on building-based pattern are treated as missing in the final data
product. For the validation of spatial signatures, this \textit{missing} category is
treated as a single class. It is assumed that the top-level large scale patterns
detected by the WorldPop method and spatial signatures will provide similar results.
However, there will be differences caused by inclusion of function in spatial
signatures, higher granularity of both initial spatial unit and the resulting
classification (6 vs 19 classes).

Signature typology is rasterized and linked to the WorldPop grid. The resulting
contingency table is shown on Figure XXX. There is a significant relationship between
two typologies, $\chi^{2} (114, N = 22993921) = 13341832, p < .001$. The strength of
association measured as Cramér's $V$ is $0.311$, indicating moderate association.
    % results + contingency table figure

\subsection*{MODUM}
% - MODUM (Spsig)
    % description of dataset + figure
    % expectations regarding similarity
    % results + contingency table figure
    chi2: 13938867.969895832, p: 0.0, dof: 152, N: 13067584
    V: 0.300


\subsection*{Copernicus Urban Atlas}
% - Urban atlas (Spsig)
    % description of dataset + figure
    % expectations regarding similarity
    % results + contingency table figure
    chi2: 5229900.007979867, p: 0.0, dof: 450, N: 8396642
    V: 0.186

\subsection*{Summary}
% Summary
    % WP
    % MODUM
    % UA