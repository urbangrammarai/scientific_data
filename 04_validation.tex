Spatial signatures are unique as a classification method, limiting the potential
validation. Therefore, we rahter present a comparison of signatures and ancillary datasets capturing
conceptually similar aspects of the environment. We compare the signatures with four of
such datasets, each focusing on a different classification perspective, but all related
to our classification to a degree when we can assume there will be a measurable level of
association between the two:

\begin{itemize}
    \item WorldPop settlement patterns of building footprints (2021)\cite{jochem2021tools}
    \item Classification of Multidimensional Open Data of Urban Morphology (MODUM) (2015)\cite{alexiou2016}
    \item Copernicus Urban Atlas (2018)\cite{eea2018}
    \item Local Climate Zones (2019)\cite{demuzere2019mapping}
\end{itemize}


\subsection*{Comparison approach}
% General method of validation
    % data transfer (one or the other way depending on feasibility) chi-squared
    % statistic Cramer's V
All datasets, spatial signatures and those selected for a comparison contain a
categorical classification of space linked to their unique geometry. The first
requirement to be able to compare data products is to transfer their
information to the same geometry. We take two approaches for this step,
depending on the dataset we are comparing the signatures with:
an interpolation of one set of polygon-based data to another (input to ETCs);
or the conversion of
spatial signatures to the raster representation matching an input raster,
which is computationally more efficient when one of the layers is already a raster. The second
step is a statistical comparison of two sets of classification labels, one representing
spatial signature typology and the other comparison classes. We use contingency tables
and Pearson's $\chi^{2}$ test to determine whether the frequencies of observed
(signature types) and expected (comparison types) labels significantly differ in one or
more categories. Furthermore, we use Cramér's $V$ statistics\cite{cramer2016mathematical} to assess the strength of
the association.

\subsection*{WorldPop settlement patterns of building footprints}
% - WorldPop (Spsig)
    % description of dataset + figure
WorldPop settlement patterns of building footprints dataset aims to derive a typology of
morphological patterns based on a gridded approach with cells of
100x100m, and building footprints. Authors measure six morphometric characters
linked to the grid cells and use them as input for an unsupervised clustering
algorithm leading to a six-class typology.
    % expectations regarding similarity
As the classification is dependent on building footprints, grid cells that do
not contain any information on the building-based pattern are treated as missing in the
final data product. For the comparison, this \textit{missing}
category is treated as a single class. It is assumed that the top-level large scale
patterns detected by the WorldPop method and spatial signatures will provide similar
results. However, there will be differences caused by the inclusion of function in spatial
signatures, higher granularity of both initial spatial units and the resulting
classification (6 vs 19 classes).

Signature typology is rasterized and linked to the WorldPop grid. The resulting
contingency table is shown in Figure \ref{fig:crosstab_worldpop}. There is a significant relationship between
two typologies, $\chi^{2} (114, N = 22993921) = 13341832, p < .001$. The strength of
association measured as Cram\'{e}r's $V$ is $0.311$, indicating moderate association.
The contingency table shows that WorldPop classes tend to be linked to groups of
signature types of a similarly degree of urbanity. A WorldPop class 15 is "undefined" due
to the lack of building footprints in the area, therefore overlapping a large portion of
signatures.
    % results + contingency table figure

\begin{figure}
    \centering
    \includegraphics[width=.8\linewidth]{fig/crosstab_worldpop.pdf}
    \caption{Contingency table showing frequencies (in \%) of WorldPop classes within signature types.}
    \label{fig:crosstab_worldpop}
\end{figure}

\subsection*{MODUM}
% - MODUM (Spsig)
    % description of dataset + figure
Multidimensional Open Data Urban Morphology (MODUM) classification describes a typology
of neighbourhoods derived from 18 indicators capturing built environment as streets,
railways or parks, linked to the Census Output Area geometry. The classification
identifies 8 types of neighbourhoods.
    % expectations regarding similarity
Compared to the WorldPop classification, MODUM takes into account more features of
the built environment than building footprints, which makes it conceptually closer to the
spatial signatures. However, it is still focusing predominantly on the form component,
although there are some indicators that would be classified as function within the
signatures framework (e.g. population). The MODUM method uses a different way of
capturing context compared to the signatures, which leads to some classes being
determined predominantly by a single character. For example, the \textit{Railway Buzz} type
forms a narrow strip around the railway network, which is an effect signatures avoid.
    % results + contingency table figure
MODUM typology is available only for England and Wales. Therefore the comparison takes
into account only ETCs covering the same area. The classification is linked to the
ETC geometry is based on the proportion (the type covering the largest portion of ETC is
assigned). The resulting contingency table is shown in Figure \ref{fig:crosstab_modum}. There is a
significant relationship between two typologies, $\chi^{2} (152, N = 13067584) =
13938867, p < .001$. The strength of association measured as Cramér's $V$ is $0.300$,
indicating moderate association of very similar levels we have seen above. The
contingency table indicates similar relationships, where a single MODUM class overlaps a
group of signature types. However, the groups tend to be well defined and formed based
on the similarity of types.

\begin{figure}
    \centering
    \includegraphics[width=.8\linewidth]{fig/crosstab_modum.pdf}
    \caption{Contingency table showing frequencies (in \%) of MODUM classes within signature types.}
    \label{fig:crosstab_modum}
\end{figure}

\subsection*{Copernicus Urban Atlas}
% - Urban atlas (Spsig)
    % description of dataset + figure
Copernicus Urban Atlas is the least similar of the comparison datasets. It is a
high-resolution land use classification of functional urban areas derived primarily from
Earth Observation data enriched by other reference data as OpenStreetMap or topographic
maps. Its smallest spatial unit in urban areas is 0.25 ha and 1 ha in rural areas,
defined primarily by physical barriers. It identifies
27 predefined classes using the supervised method.
    % expectations regarding similarity
The majority of urban areas is classified as urban fabric further distinguished based on
continuity and density resulting in six classes of the urban fabric. The classification does
not consider the type of the pattern or any other aspect. Furthermore, it does not take
into account what signatures call \textit{context} as each spatial unit is
classified independently, which in some cases leads to the high heterogeneity of
classification within a small portion of land. Signatures take a different approach.
Consequently, it is expected that the similarity between the two will be limited.
    % results + contingency table figure
Urban Atlas is available only for functional urban areas (FUA), leaving rural areas
unclassified. Comparison then applies to FUAs only. The classification is linked to the
ETC geometry based on the proportion (the type covering the largest portion of ETC is
assigned). The resulting contingency table is shown in Figure \ref{fig:crosstab_ua}. There is a
significant relationship between two typologies, $\chi^{2} (450, N = 8396642) = 5229900,
p < .001$. The strength of association measured as Cramér's $V$ is $0.186$, indicating
a weak association. The contingency table shows the difference in the aim of spatial
signatures and that of Urban Atlas with a majority of signatures being linked to a few
of Urban Atlas classes. Within relevant classes, we see a tendency of signature types to
cluster within Urban Atlas classes based on the level of urbanity, albeit not as strong
as in the previous two cases.

\begin{figure}
    \centering
    \includegraphics[width=\linewidth]{fig/crosstab_ua.pdf}
    \caption{Contingency table showing frequencies (in \%) of Urban Atlas classes within signature types.}
    \label{fig:crosstab_ua}
\end{figure}

\subsection*{Local Climate Zones}
% - Local climate zones (Spsig)
    % description of dataset + figure
Local climate zones (LCZ) are conceptual classes originally designed to support study of urban
climate as temperature. It consists of 17 classes of which 10 can be classified as urban
and 7 and natural ones. In the context of Great Britain, the dataset used in this study
does not contain 2 of them, \textit{Lightweight low-rise} and \textit{Compact highrise}
as they are not present in the British landscape. The datasets produced by
\cite{demuzere2019mapping} released LCZs in a 100 meters grid based on the 2016 data. As
the LCZs are remotely sensed in this case, authors report overall average accuracy of 80 \%.
    % expectations regarding similarity
As a conceptual classification aimed to cover all possible types of primariliy urban climate zones globally,
LZCs may not be optimal when looking into a single country with specific history of urban
development. This is furhter indicated by classes that are missing. It is therefore likely
that large parts of British cities will fall into only a few of LCZ classes, while being representated
by a much larger number of signature types.
    % results + contingency table figure

Signature typology is rasterized and linked to the LCZ grid.
The resulting contingency table is shown in Figure \ref{fig:crosstab_lcz}. There is a
significant relationship between two typologies, $\chi^{2} (225, N = 16203338) = 18467242,
p < .001$. The strength of association measured as Cramér's $V$ is $0.276$, indicating
a modest to weak association, close to values we've seen in first two cases. As expected,
urban signature types are clustered primarily within \textit{Compact midrise} and
\textit{Open lowrise} LCZs, while non-urban signatures mostly fall into the \textit{Low plants} LCZ.

\begin{figure}
    \centering
    \includegraphics[width=.8\linewidth]{fig/crosstab_lcz.pdf}
    \caption{Contingency table showing frequencies (in \%) of Local Climate Zones within signature types.}
    \label{fig:crosstab_lcz}
\end{figure}

\subsection*{Summary}
None of the comparisons shows more than a moderate association, but since none of the
comparison datasets is aiming to capture the same conceptualization of space as spatial
signatures do, such a result is expected. The moderate association with both WorldPop
settlements patterns and MODUM is reassuring as both are conceptually closer to
signatures than the Urban Atlas (especially in their unsupervised design). Urban Atlas,
though very different in its aims and methods, still shows a measurable association,
which we interpret as sign that the key structural aspects forming cities are captured by both. The
comparison exercise suggests that general patterns forming cities are shared among
signatures and existing typologies. Signature types tend to form groups when we look at
their relation to comparison classes and it is not uncommon that a single signature type
is present in multiple groups linked to different classes. However, all these groups
tend to be formed based on the similarity and illustrate the granularity of the
presented classification compared to existing datasets, allowing us to distinguish, for
example, five types of signature types forming town an city centres.

