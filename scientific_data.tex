\documentclass[fleqn,10pt]{wlscirep}
\usepackage[utf8]{inputenc}
\usepackage[T1]{fontenc}
\usepackage{lineno}
\linenumbers

\title{Geographical Characterisation of British Urban Form and Function using
the Spatial Signatures Framework}

\author[1, *]{Martin Fleischmann}
\author[1]{Daniel Arribas-Bel}
\affil[1]{Geographic Data Science Lab, Department of Geography and Planning, University
of Liverpool, Roxby Building , 74 Bedford St S , Liverpool , L69 7ZT, United Kingdom}

\affil[*]{corresponding author(s): Martin Fleischmann (m.fleischmann@liverpool.ac.uk)}


\begin{abstract}
% This is a manuscript template for Data Descriptor submissions to \emph{Scientific Data}
% (\href{http://www.nature.com/scientificdata}{http://www.nature.com/scientificdata}). The
% abstract must be no longer than 170 words, and should succinctly describe the study, the
% assay(s) performed, the resulting data, and the reuse potential, but should not make any
% claims regarding new scientific findings. No references are allowed in this section.

% intro
The spatial arrangement of the building blocks that make up cities matters  to
understand the rules directing their dynamics.
% one line summary
Our study outlines the development of the national open-source classification of space
according to its form and function into a single typology.
% input
We create a bespoke granular spatial unit, enclosed tessellation, and measure
characters capturing its form and function within a relevant spatial context.
% method
Using the two-level K-Means clustering of individual enclosed tessellation cells, we generate
classification of space for the whole of Great Britain.
% result
Contiguous enclosed tessellation cells belonging to the same class are merged
forming spatial signature geometries and their typology.
% findings
We identify 16 distinct types of spatial signatures stretching from a wild countryside,
through various kinds of suburbia to types denoting urban centres according to their
regional importance.
% use case
The open data product has the potential to serve as a boundary delineation for other
researchers interested in urban environment and policymakers looking for a unique
perspective on cities and their structure.

\end{abstract}
\begin{document}

\flushbottom
\maketitle
%  Click the title above to edit the author information and abstract

\thispagestyle{empty}

% \noindent Please note: Abbreviations should be introduced at the first mention in the
% main text – no abbreviations lists or tables should be included. Structure of the main
% text is provided below.

\section*{Background \& Summary}

%       (700 words maximum) An overview of the study design, the assay(s) performed, and the
%       created data, including any background information needed to put this study in the
%       context of previous work and the literature. The section should also briefly outline the
%       broader goals that motivated the creation of this dataset and the potential reuse value.
%       We also encourage authors to include a figure that provides a schematic overview of the
%       study and assay(s) design. The Background \& Summary should not include subheadings.
%       This section and the other main body sections of the manuscript should include citations
%       to the literature as needed.

% This section should provide an overview of the study that generated the data, as well as outlining the potential reuse value of the data. Any previous publications that used these data, in whole or in part, should be cited and briefly summarized.

% + Form vs function in classification & missing link
% + detailed x scalable x consistent
% + *largely rephrase main arguments from conceptual paper*
% + summary - spsig in the whole GB
%     - classification and description of space


% Form & Function in cities and their value
How the building blocks that make up cities are spatially arranged is worth
quantifying and understanding.
%% What FF is
By "building blocks", we mean both the activities and agents that inhabit
cities, as well as the (infra)structure that supports them. The
former can be conceptualised as \textit{urban function}, while the latter
falls under the study of \textit{urban form}.
%% Why:
Understanding urban form and function is important for two main reasons.
%%% encodes
First, the combination of both \textit{encodes} much information about the
history, character and evolution of cities.
%
For example, the shape and properties of the street network encode the technology of the
time (e.g., automobile); while the degree of mix in land uses can reflect
cultural values.
%%% influences
Second, the spatial pattern of urban form and function also acts as a
frame that \textit{influences} a variety of outcomes, from economic
productivity to socio-economic cohesion to environmental sustainability.

% We use the Spatial Signatures
In this paper, we use the Spatial Signatures framework \cite{dab_mf_2021a, dab_mf_2021b},
which develops a ``characterisation of space based on form and function
designed to understand urban environments''\cite{dab_mf_2021a}.
%% Signature definition
Spatial Signatures are theory-informed, data-driven computable classes that
describe the form and function of a consistent patch of geography.
%
Figure \ref{fig:workflow} presents an overview of the development of a spatial
signature classification.
%
We build a series of enclosures that we combine with building footprints
to further subdivide geographical space. We then attach form and function
characters to each of these subdivisions, and use those to group them into
consistent and differentiated classes we call signatures.
%
Each phase is expanded in detail in the next section.

\begin{figure}
        \includegraphics[width=\linewidth]{fig/workflow.pdf}
\caption{Diagram illustrating the sequential steps leading to the delineation of
spatial signatures. From a series of enclosing components, to enclosures,
enclosed tessellation (ET), the addition of form and function characters to ET
cells, and the development of spatial signatures.}
\label{fig:workflow}
\end{figure}

    % In this paper
% Great British Signatures
We introduce an open data product (ODP \cite{odp_paper21}) containing a classification of
spatial signatures for Great Britain. In doing so, we provide an
analysis-ready layer that brings urban form and function consistently, in
detail, and at national scale. To the best of our knowledge, this is the first
dataset capturing urban form and function published both with a degree of detail and scale as
ours.
%% Input data used
Our results are based on the analysis of more than 14 million of ET cells, to
each of which we attach more than 300 characters capturing a wide range of
aspects relating to urban form and function.
%% Data available
We provide access to both granular geographical boundaries of the delineated spatial signatures
as well as measurements for each character at the signature level.
%% Web map and code
The ODP also includes a web map that allows exploration without any technical
requirement other than a web browser, and we have open sourced all the code,
including details on the computational backend.
%% Comparison with other datasets
The uniqueness of our ODP makes setting up a technical validation as a
comparison with existing datasets challenging. Nevertheless, we relate our
signatures to a few well-established data products that capture each a subset
of the form and function dimensions we consider. Our results are encouraging
in that they show broad agreement in expected areas, but also highlight
aspects that can only be discovered when considering form and function in tandem.

% Benefits of signature approach (goals why we created it)
The approach and outputs presented bring several benefits to a range of
stakeholders interested in cities.
%% ODP
This spatial signatures ODP provides insight generated from detailed,
comprehensive and computationally intensive data analysis and presents it in a
way that is easy to access, work with and integrate into larger projects.
%
%%% Academics and policymakers
Together with the importance of form and function discussed above, we
anticipate the output will be relevant to both academic researchers as well as
policymakers and practitioners.
%% Broader SS benefits
As a conceptual framework, the spatial signatures provide a flexible yet
generalisable approach to understand, characterise and quantify urban form and
function. One way to understand our results is as an implementation of a
more general way of thinking about the spatial dimension of cities. In this
context, it can be useful to researchers and practitioners who, even if not
specifically interested in Great Britain, would like to implement a similar
approach.
%
In this respect, we hope the present paper serves not only to document our own
work but to inspire future efforts aimed at urban form and function.



            %%% FROM ORIGNAL GUIDELINE %%%

% An overview of the study design, the assay(s) performed, and the created data, including any background information needed to put this study in the context of previous work and the literature.

%- Briefly outline the broader goals that motivated the creation of this dataset and the potential reuse value.

%- We also encourage authors to include a figure that provides a schematic overview of the study and assay(s) design.




\section*{Methods}

% conceptualisation of signature detection
    % small unit
    % complex descirption (F&F)
    % cluster analysis
The method of identification of spatial signatures consists of three top level steps.
First, we need to delineate spatial unit of analysis, one that reflects the structure of
urban phenomena on a very granular level. Then we charactercterise each of them
according to the form and function capturing the nature of each unit and its spatial
context. Finally, we use cluster analysis to derive a typology of our spatial units
that, once combined into contiguous areas, forms a typology of spatial signatures.

\subsection*{Spatial unit}
% spatial unit
    % conceptualisation of enclosed tessellation
    % rules
        % indivisible
        % internally consistent
        % geographically exhaustive
    % options
        % admin boundaries
        % arbitrary grids
        % morphological units
    % ET design
        % barriers
        % enclosures
        % anchors
        % ET cells
The first major methodological decision needs to be taken on the definition of the
spatial unit. As mentioned, it needs to reflect space in a granular manner and we argue
that it should fulfil three conditions. First, it should be \textit{indivisible},
meaning that when such a unit would be subdivided into smaller parts, none of them would
be enough to capture the nature of spatial signature. Second, it needs to be
\textit{internally consistent} - it should always reflect only a single signature type.
Last, it should be geographically \textit{exhaustive}, covering entirety of the study
area.

Spatial units used in literature can be split into three groups. One is using
administrative boundaries like city regions, wards or census output areas, that are
convenient to obatain and can be easily linked to auxiliarry data. However,
those rarely reflect the morphological composition of urban space and in some cases may
even “obscure morphologic reality” REF Taubenbock 2019. At the same time, most of them
are divisible and larger units are not alwyas internally consistent. Another group is based on
arbitrary uniform grids linked either to spatial indexing method like H3 REF or OS
National Grid REF, or to anciliarry data of remote sensing or other origins like a
WorldPop grid REF. The issue is that grids cannot be considered internally consistent as
they have no relation to the real-life spatial pattern. Finally, urban morphology tends to use morphological elements as
street segments REF, blocks REF buildings or plots as a unit of analysis. Some of those
could be seen as indivisible and internally consistent but since they are largely based
on built-up fabric, they are not exhaustive. When there is no builiding or street, there
is no spatial unit to work with. Plots could be theoretically considered as exhaustive,
consistent and indivisible but there is no accepted conceptual definition and unified
geometric representation (REF Kropf).

We are, therefore, proposing an application of an alternative spatial unit called \textit{enclosed
tessellation cell} (ETC), defined as:

\newtheorem*{theorem}{}
\begin{theorem}
    A characterisation of space based on form and function designed to understand urban
environments
\end{theorem}

% We should drop a reference to conceptual paper here.

ETC follows the morphological tradition in a sense that is it
based on the physical elements of an enviroment but overcomes the drawbacks of
conventionally used units. Its geometry is generated in three steps illustrated on a
Figure \ref{fig:et_diagram}. First, a set of features representing physical barriers
subdividing space, in our case composed of street network, railways, rivers and a
coastline, is combined together, generating a layer of boundaries. These then partition space
into smaller enclosed geometries called \textit{enclosures}, which can be very granular
or very coarse depending on the geographic context. In dense city centres where a single
enclosure represent a single block is a high frequency of small enclosures, while in the
countryside, we can observe very few large enclosures as their delimiters are far away
from each other. Enclosures are then combined with building footprints, posing as
anchors in the space and are subdivided into enclosed tessellation cells using the
morphological tessellation algorithm REF, a polygon-based adaptation of Voronoi
tessellation. Resulting geometries are indivisible as they contain, at most, a single
achor building, internally consistent due to their granularity and link to morphological
elements composing urban fabric, and geogrpahically exhasutive as they cover entire area
limited by specified boundaries.

    % data input
        % barries
            % roads from OS Open Roads
                % data description (simplified centerlines)
            % railway from OS OpenMap - Local
                % data description (simplified centerlines)
            % rivers from OS OpenRivers
                % data description (simplified centerlines)
            % coastline from OS Strategi®
                % data description (continuous enclosing geometry)
        % buildings
            % OS OpenMap - Local
            % data quality description (merged adjacent geometries)

In the case of classification of Great Britain, street networks are extracted from OS
Open Roads datasets (REF) representing simplified road centrelines cleaned of road
segments under the ground. Railways are retrieved from OS OpenMap - Local
("RailwayTrack" layer) which captures surfance railway tracks. Rivers are extracted from
OS OpenRivers (REF) representing river network of GB as centrelines, and a coastline is
retrieved from OS Strategi® (2016) REF, capturing  coastline as a continous line
geometry. Building geometry is extracted, again, from OS OpenMap - Local ("Building"
layer) and represents generalised building footprint polygons. Note that the dataset
does not distinguish between individual buildings when they are adajcent (e.g. perimeter
block composed of multiple buildings is represent by a single polygon).

% characterisation of space
    % form
        % input data
            % ET cells
            % bulidings
            % street network
        % morphometrics
            % different categories of characters
                % dimension
                % shape
                % spatial distribution
                % intensity
                % connectivity
                % diversity
            % different scales
                % individual elements -> adjacency -> networks
        % contextualisation
            % interest in characterisation of spatial patterns
            % distance-weighted higher order contiguity spatial weights
            % reflection of a statistical distribution of data within context
                % proxy of diversity

    % function
        % input data
            % overview - from population and POIs to NDVI and night lights
            % include table with transfer methods
        % transfer methods overview
            % spatial join
            % interpolation
            % accessibility
        % contextualisation
            % depending on the transfer method, function-based characters we
            %  contextualised using the same method used in morphometrics

% cluster analysis
    % comparison of clustering methods (shall we include this?? skip for now)
        % K-Means, GMM, SOM

    % two levels of K-Means clustering
        % data input
            % combination of both form and function, all characters equally weighted
            % only contextualised representation of characters is used to capture pattern

            % data standardisation
                % column-based mean standardisation

        % selection of number of clusters
            % clustergram
            % supplementary metrics
                % Silhouette
                % Calinski-Harabasz
                % Davies-Boulding

        % top level providing a first national classification
        % sub clustering of urban areas
            % selection criteria for to-be-subclustered classes

% generation of signature geometry
    % dissolution based on contiguity and assigned class


% - Feature importance - should this actually be included in here?

\section*{Data Records}

% The Data Records section should be used to explain each data record associated with this
% work, including the repository where this information is stored, and to provide an
% overview of the data files and their formats. Each external data record should be cited
% numerically in the text of this section, for example \cite{Hao:gidmaps:2014}, and
% included in the main reference list as described below. A data citation should also be
% placed in the subsection of the Methods containing the data-collection or analytical
% procedure(s) used to derive the corresponding record. Providing a direct link to the
% dataset may also be helpful to readers
% (\hyperlink{https://doi.org/10.6084/m9.figshare.853801}{https://doi.org/10.6084/m9.figshare.853801}).

% Tables should be used to support the data records, and should clearly indicate the
% samples and subjects (study inputs), their provenance, and the experimental
% manipulations performed on each (please see 'Tables' below). They should also specify
% the data output resulting from each data-collection or analytical step, should these
% form part of the archived record.

\section{Data Records}
\label{sec:data}

% description of the GPKG/archive

\section*{Technical Validation}

% This section presents any experiments or analyses that are needed to support the
% technical quality of the dataset. This section may be supported by figures and tables,
% as needed. This is a required section; authors must present information justifying the
% reliability of their data.

\subsection*{Character importance}

The characters used in the cluster analysis have each different importance in distinguish between signature types.
Those characters which spatial distribution most closely matches the distribution of signatures
can be seen as more important that those that are seemingly random or mostly invariant (as some of the land cover classes are).
Unpacking the importance of individual characters from K-Means clustering cannot be done directly, but
a useful method is to train a supervised model, in our case Random Forest, designed to predict individual
signature types from input data. Such a model then provides a feature importance - a relative measure of
a strenght of each character in distinguishing between the types. The results of this approach are shown in
a table \ref{tab:imp}. As you can see, form-based characters dominate the top 10 characters but it is worth
noting that these top 10 characters together bear only 0.196 of the overall importance.

\begin{table}
\begin{tabular}{lr}
    \toprule
    {} &  relative importance \\
    \midrule
    covered area ratio of ETC (Q1)             &             0.036944 \\
    covered area ratio of ETC (Q2)             &             0.031717 \\
    perimeter-weighted neighbours of ETC (Q2)  &             0.023476 \\
    mean inter-building distance (Q2)          &             0.016662 \\
    area of ETC (Q3)                           &             0.016005 \\
    area covered by node-attached ETCs (Q3)    &             0.014813 \\
    longest axis length of ETC (Q2)            &             0.014501 \\
    weighted reached enclosures of ETC (Q1)    &             0.014115 \\
    reached area by neighbouring segments (Q3) &             0.014000 \\
    reached area by neighbouring segments (Q1) &             0.013904 \\
    \bottomrule
\end{tabular}
\caption{\label{tab:imp}Relative importance of top 10 most important characters in
predicting spatial signature types using the Random Forest model.}
\end{table}

A similar exercise can be done on a level of individual clusters, with a binary Random Forest model trained
to distingish that particular class from the other. Resulting relative importance of top 10 characters for each signature type
is presented in a table \ref{tab:imp_cls}. While it is clear that form-based characters still dominate the prediction,
the more urban signature types are, the higher the importance of function seems to be. Complete tables
with all characters are availble as online tables 1 and 2.

\begin{table}
    \begin{tabular}{lrlrlrlrlrlrlrlrlrlrlrlrlrlrlrlr}
        \toprule
                                         Wild countryside & \multicolumn{2}{l}{Countryside agriculture} & \multicolumn{2}{l}{Gridded residential quarters} & \multicolumn{2}{l}{Accessible suburbia} & \multicolumn{2}{l}{Connected residential neighbourhoods} & \multicolumn{2}{l}{Urban buffer} & \multicolumn{2}{l}{Open sprawl} & \multicolumn{2}{l}{Warehouse/Park land} & \multicolumn{2}{l}{Local urbanity} & \multicolumn{2}{l}{Dense residential neighbourhoods} & \multicolumn{2}{l}{Disconnected suburbia} & \multicolumn{2}{l}{Dense urban neighbourhoods} & \multicolumn{2}{l}{Regional urbanity} & \multicolumn{2}{l}{Metropolitan urbanity} & \multicolumn{2}{l}{Concentrated urbanity} & \multicolumn{2}{l}{Hyper concentrated urbanity} \\
                                                     name & rel. importance &                                               name & rel. importance &                                               name & rel. importance &                                         name & rel. importance &                                               name & rel. importance &                                               name & rel. importance &                                        name & rel. importance &                                               name & rel. importance &                                               name & rel. importance &                                             name & rel. importance &                                               name & rel. importance &                                               name & rel. importance &                                               name & rel. importance &                                               name & rel. importance &                                               name & rel. importance &                                               name & rel. importance \\
        \midrule
                          longest axis length of ETC (Q1) &        0.196609 &                     covered area ratio of ETC (Q1) &        0.154212 &             local closeness of street network (Q3) &        0.095416 &      weighted reached enclosures of ETC (Q3) &        0.063690 &                    cell alignment of building (Q1) &        0.028481 &            area covered by neighbouring cells (Q2) &        0.071780 &   reached area by local street network (Q1) &        0.058349 &                        elongation of building (Q1) &        0.033986 &                         perimeter of building (Q2) &        0.101218 & centroid - corner mean distance of building (Q2) &        0.036962 & local proportion of cul-de-sacs of street netwo... &        0.023643 &                         perimeter of building (Q2) &        0.106663 & centroid - corner distance deviation of buildin... &        0.115378 &      equivalent rectangular index of building (Q2) &        0.110605 &                              area of building (Q1) &        0.128233 &                     covered area ratio of ETC (Q2) &        0.154124 \\
                           covered area ratio of ETC (Q2) &        0.151118 &                     covered area ratio of ETC (Q2) &        0.144279 &             local closeness of street network (Q2) &        0.046007 & reached ETCs by tessellation contiguity (Q3) &        0.062061 & local proportion of 4-way intersections of stre... &        0.022801 &                     covered area ratio of ETC (Q2) &        0.049986 &  reached area by neighbouring segments (Q1) &        0.033786 &   centroid - corner mean distance of building (Q3) &        0.028038 &      equivalent rectangular index of building (Q1) &        0.093518 & centroid - corner mean distance of building (Q3) &        0.030172 &            local meshedness of street network (Q3) &        0.021436 &   centroid - corner mean distance of building (Q2) &        0.084484 &   centroid - corner mean distance of building (Q2) &        0.088275 &   centroid - corner mean distance of building (Q2) &        0.087236 & Workplace population [Distribution, hotels and ... &        0.099872 &          Workplace population [Manufacturing] (Q2) &        0.143669 \\
                           covered area ratio of ETC (Q1) &        0.145754 &                  mean inter-building distance (Q2) &        0.078928 &                        perimeter of enclosure (Q1) &        0.044369 & reached area by tessellation contiguity (Q2) &        0.047628 &                    cell alignment of building (Q2) &        0.017361 & mean distance to neighbouring nodes of street n... &        0.048812 &     area covered by node-attached ETCs (Q2) &        0.023548 &                        elongation of building (Q2) &        0.025310 &   centroid - corner mean distance of building (Q2) &        0.082040 &                            area of building (Q3) &        0.029425 &            local meshedness of street network (Q2) &        0.020520 &                         perimeter of building (Q3) &        0.082203 &                        squareness of building (Q3) &        0.082130 & centroid - corner distance deviation of buildin... &        0.080684 & Workplace population [Financial, real estate, p... &        0.076751 &                  Workplace population [Other] (Q2) &        0.101678 \\
                                         area of ETC (Q2) &        0.096485 &                                   area of ETC (Q2) &        0.072977 &                             area of enclosure (Q2) &        0.037296 &                             area of ETC (Q2) &        0.045100 &                             area of enclosure (Q2) &        0.017232 &                     covered area ratio of ETC (Q1) &        0.046002 &              covered area ratio of ETC (Q2) &        0.022004 &              circular compactness of building (Q1) &        0.019596 &                        squareness of building (Q3) &        0.054375 &                                  Population (Q3) &        0.028051 &      equivalent rectangular index of building (Q1) &        0.020273 &                              area of building (Q2) &        0.066331 & Workplace population [Financial, real estate, p... &        0.071195 &                           corners of building (Q2) &        0.071910 &                  Workplace population [Other] (Q2) &        0.075844 & Workplace population [Distribution, hotels and ... &        0.081796 \\
                perimeter-weighted neighbours of ETC (Q3) &        0.075078 &            area covered by node-attached ETCs (Q2) &        0.066916 &             local closeness of street network (Q1) &        0.036698 &   reached ETCs by neighbouring segments (Q1) &        0.036865 &                            orientation of ETC (Q2) &        0.016519 &         reached area by neighbouring segments (Q1) &        0.038292 &   local node density of street network (Q3) &        0.018821 & centroid - corner distance deviation of buildin... &        0.017568 &                              area of building (Q2) &        0.051383 &                       perimeter of building (Q2) &        0.025995 &              circular compactness of building (Q1) &        0.019361 &                                    Population (Q3) &        0.039538 &                         perimeter of building (Q2) &        0.065223 & Workplace population [Financial, real estate, p... &        0.060205 & Workplace population [Distribution, hotels and ... &        0.070516 &                     covered area ratio of ETC (Q1) &        0.079165 \\
               reached area by neighbouring segments (Q1) &        0.048869 & mean distance to neighbouring nodes of street n... &        0.066150 &            weighted reached enclosures of ETC (Q3) &        0.031708 &   reached ETCs by neighbouring segments (Q2) &        0.029904 &      equivalent rectangular index of building (Q1) &        0.016281 &                   circular compactness of ETC (Q2) &        0.034693 &  reached area by neighbouring segments (Q2) &        0.018065 &                         perimeter of building (Q3) &        0.017147 & centroid - corner distance deviation of buildin... &        0.044975 &                            area of building (Q2) &        0.022935 &                                    Population (Q1) &        0.017855 &                        squareness of building (Q3) &        0.039269 &                         perimeter of building (Q3) &        0.057863 & Workplace population [Distribution, hotels and ... &        0.051041 & Workplace population [Financial, real estate, p... &        0.060396 &          Workplace population [Manufacturing] (Q3) &        0.074537 \\
             reached area by tessellation contiguity (Q1) &        0.018289 &         reached area by neighbouring segments (Q1) &        0.062776 & local proportion of 4-way intersections of stre... &        0.020617 &    reached ETCs by local street network (Q2) &        0.026220 & local proportion of 4-way intersections of stre... &        0.014317 &            area covered by neighbouring cells (Q1) &        0.033082 &              covered area ratio of ETC (Q1) &        0.018036 &                       width of street profile (Q2) &        0.016790 & Workplace population [Financial, real estate, p... &        0.044019 &                      perimeter of enclosure (Q1) &        0.020672 &                        elongation of building (Q2) &        0.016115 & centroid - corner distance deviation of buildin... &        0.034407 &                              area of building (Q2) &        0.050349 &                         perimeter of building (Q2) &        0.046511 &          Workplace population [Manufacturing] (Q2) &        0.054662 &   centroid - corner mean distance of building (Q2) &        0.070363 \\
                                         area of ETC (Q3) &        0.015991 &       Land cover [Discontinuous urban fabric] (Q2) &        0.055444 &            area covered by node-attached ETCs (Q1) &        0.019211 &    perimeter-weighted neighbours of ETC (Q1) &        0.023718 &                        perimeter of enclosure (Q1) &        0.014305 &         buildings per meter of street segment (Q2) &        0.032175 &                      area of enclosure (Q2) &        0.017156 &              circular compactness of building (Q2) &        0.016306 & Workplace population [Distribution, hotels and ... &        0.035144 &                    orientation of enclosure (Q2) &        0.017636 &         reached area by neighbouring segments (Q2) &        0.015907 & Workplace population [Financial, real estate, p... &        0.029293 & Workplace population [Distribution, hotels and ... &        0.048963 &                        squareness of building (Q3) &        0.039035 &                         perimeter of building (Q2) &        0.046941 &                         perimeter of building (Q2) &        0.054542 \\
        mean distance between neighbouring buildings (Q2) &        0.015013 &          perimeter-weighted neighbours of ETC (Q2) &        0.021903 &            area covered by node-attached ETCs (Q2) &        0.018326 & reached area by tessellation contiguity (Q1) &        0.022821 & local proportion of cul-de-sacs of street netwo... &        0.013748 &       reached area by tessellation contiguity (Q1) &        0.029531 & compactness-weighted axis of enclosure (Q3) &        0.016709 &       reached area by tessellation contiguity (Q1) &        0.016248 &                         perimeter of building (Q3) &        0.033695 &                       perimeter of building (Q3) &        0.016561 &            area covered by edge-attached ETCs (Q3) &        0.015526 &      equivalent rectangular index of building (Q1) &        0.017548 &                           corners of building (Q3) &        0.029268 & Workplace population [Financial, real estate, p... &        0.029708 &   centroid - corner mean distance of building (Q2) &        0.044694 &                    openness of street profile (Q2) &        0.031258 \\
                        mean inter-building distance (Q2) &        0.010559 &                    longest axis length of ETC (Q2) &        0.020504 &            weighted reached enclosures of ETC (Q2) &        0.016607 &    reached ETCs by local street network (Q1) &        0.020275 &                      orientation of enclosure (Q1) &        0.013305 &            area covered by node-attached ETCs (Q3) &        0.027952 &                            area of ETC (Q2) &        0.015707 &                         perimeter of building (Q2) &        0.014735 &                              area of building (Q1) &        0.022769 &                           area of enclosure (Q1) &        0.015454 &              circular compactness of building (Q2) &        0.015257 &                  Workplace population [Other] (Q2) &        0.016284 & centroid - corner distance deviation of buildin... &        0.020625 &   centroid - corner mean distance of building (Q1) &        0.018957 &        Land cover [Non-irrigated arable land] (Q1) &        0.025912 &                                          NDVI (Q3) &        0.027406 \\
        \bottomrule
        \end{tabular}
        \caption{\label{tab:imp_cls}Relative importance of top 10 most important characters for each signature type in
        predicting using the Random Forest model.}
\end{table}

\subsection*{Comparison}

Spatial signatures are unique as a classification method, limiting the potential
validation. Therefore, we rahter present a comparison of signatures and ancillary datasets capturing
conceptually similar aspects of the environment. We compare the signatures with four of
such datasets, each focusing on a different classification perspective, but all related
to our classification to a degree when we can assume there will be a measurable level of
association between the two:

\begin{itemize}
    \item WorldPop settlement patterns of building footprints (2021)\cite{jochem2021tools}
    \item Classification of Multidimensional Open Data of Urban Morphology (MODUM) (2015)\cite{alexiou2016}
    \item Copernicus Urban Atlas (2018)\cite{eea2018}
    \item Local Climate Zones (2019)\cite{demuzere2019mapping}
\end{itemize}


\subsection*{Comparison approach}
% General method of validation
    % data transfer (one or the other way depending on feasibility) chi-squared
    % statistic Cramer's V
All datasets, spatial signatures and those selected for a comparison contain a
categorical classification of space linked to their unique geometry. The first
requirement to be able to compare data products is to transfer their
information to the same geometry. We take two approaches for this step,
depending on the dataset we are comparing the signatures with:
an interpolation of one set of polygon-based data to another (input to ETCs);
or the conversion of
spatial signatures to the raster representation matching an input raster,
which is computationally more efficient when one of the layers is already a raster. The second
step is a statistical comparison of two sets of classification labels, one representing
spatial signature typology and the other comparison classes. We use contingency tables
and Pearson's $\chi^{2}$ test to determine whether the frequencies of observed
(signature types) and expected (comparison types) labels significantly differ in one or
more categories. Furthermore, we use Cramér's $V$ statistics\cite{cramer2016mathematical} to assess the strength of
the association.

\subsection*{WorldPop settlement patterns of building footprints}
% - WorldPop (Spsig)
    % description of dataset + figure
WorldPop settlement patterns of building footprints dataset aims to derive a typology of
morphological patterns based on a gridded approach with cells of
100x100m, and building footprints. Authors measure six morphometric characters
linked to the grid cells and use them as input for an unsupervised clustering
algorithm leading to a six-class typology.
    % expectations regarding similarity
As the classification is dependent on building footprints, grid cells that do
not contain any information on the building-based pattern are treated as missing in the
final data product. For the comparison, this \textit{missing}
category is treated as a single class. It is assumed that the top-level large scale
patterns detected by the WorldPop method and spatial signatures will provide similar
results. However, there will be differences caused by the inclusion of function in spatial
signatures, higher granularity of both initial spatial units and the resulting
classification (6 vs 19 classes).

Signature typology is rasterized and linked to the WorldPop grid. The resulting
contingency table is shown in Figure \ref{fig:crosstab_worldpop}. There is a significant relationship between
two typologies, $\chi^{2} (114, N = 22993921) = 13341832, p < .001$. The strength of
association measured as Cram\'{e}r's $V$ is $0.311$, indicating moderate association.
The contingency table shows that WorldPop classes tend to be linked to groups of
signature types of a similarly degree of urbanity. A WorldPop class 15 is "undefined" due
to the lack of building footprints in the area, therefore overlapping a large portion of
signatures.
    % results + contingency table figure
The difference between classifications is likely driven by two main aspects - one is the different
number of classes. We can see that WorldPop classes tend to cluster wihtin a limited number of
signature types and vice versa. The only exception is allocation of signature types into classes 4 and 6,
which seems to heavily overlap. That is possibly caused by the second aspect - inclusion of function. Both
classes 4 and 6 tend to be outside of city centres but still within urban areas. While it is
the footprint-based form that is driving the difference between them, signatures in the same
area are often disntiguished by function and varies access to amenities and services.

\begin{figure}
    \centering
    \includegraphics[width=.8\linewidth]{fig/crosstab_worldpop.pdf}
    \caption{Contingency table showing frequencies (in \%) of WorldPop classes within signature types.}
    \label{fig:crosstab_worldpop}
\end{figure}

\subsection*{MODUM}
% - MODUM (Spsig)
    % description of dataset + figure
Multidimensional Open Data Urban Morphology (MODUM) classification describes a typology
of neighbourhoods derived from 18 indicators capturing built environment as streets,
railways or parks, linked to the Census Output Area geometry. The classification
identifies 8 types of neighbourhoods.
    % expectations regarding similarity
Compared to the WorldPop classification, MODUM takes into account more features of
the built environment than building footprints, which makes it conceptually closer to the
spatial signatures. However, it is still focusing predominantly on the form component,
although there are some indicators that would be classified as function within the
signatures framework (e.g. population). The MODUM method uses a different way of
capturing context compared to the signatures, which leads to some classes being
determined predominantly by a single character. For example, the \textit{Railway Buzz} type
forms a narrow strip around the railway network, which is an effect signatures avoid.
    % results + contingency table figure
MODUM typology is available only for England and Wales. Therefore the comparison takes
into account only ETCs covering the same area. The classification is linked to the
ETC geometry is based on the proportion (the type covering the largest portion of ETC is
assigned). The resulting contingency table is shown in Figure \ref{fig:crosstab_modum}. There is a
significant relationship between two typologies, $\chi^{2} (152, N = 13067584) =
13938867, p < .001$. The strength of association measured as Cramér's $V$ is $0.300$,
indicating moderate association of very similar levels we have seen above. The
contingency table indicates similar relationships, where a single MODUM class overlaps a
group of signature types. However, the groups tend to be well defined and formed based
on the similarity of types. Signature types are minimally present in MODUM classes driven
by a single character (\textit{Railway Buzz}, \textit{Waterside Settings},
\textit{High Street and Promenades}), suggesting the more balanced weight of characters.


\begin{figure}
    \centering
    \includegraphics[width=.8\linewidth]{fig/crosstab_modum.pdf}
    \caption{Contingency table showing frequencies (in \%) of MODUM classes within signature types.}
    \label{fig:crosstab_modum}
\end{figure}

\subsection*{Copernicus Urban Atlas}
% - Urban atlas (Spsig)
    % description of dataset + figure
Copernicus Urban Atlas is the least similar of the comparison datasets. It is a
high-resolution land use classification of functional urban areas derived primarily from
Earth Observation data enriched by other reference data as OpenStreetMap or topographic
maps. Its smallest spatial unit in urban areas is 0.25 ha and 1 ha in rural areas,
defined primarily by physical barriers. It identifies
27 predefined classes using the supervised method.
    % expectations regarding similarity
The majority of urban areas is classified as urban fabric further distinguished based on
continuity and density resulting in six classes of the urban fabric. The classification does
not consider the type of the pattern or any other aspect. Furthermore, it does not take
into account what signatures call \textit{context} as each spatial unit is
classified independently, which in some cases leads to the high heterogeneity of
classification within a small portion of land. Signatures take a different approach.
Consequently, it is expected that the similarity between the two will be limited.
    % results + contingency table figure
Urban Atlas is available only for functional urban areas (FUA), leaving rural areas
unclassified. Comparison then applies to FUAs only. The classification is linked to the
ETC geometry based on the proportion (the type covering the largest portion of ETC is
assigned). The resulting contingency table is shown in Figure \ref{fig:crosstab_ua}. There is a
significant relationship between two typologies, $\chi^{2} (450, N = 8396642) = 5229900,
p < .001$. The strength of association measured as Cramér's $V$ is $0.186$, indicating
a weak association. The contingency table shows the difference in the aim of spatial
signatures and that of Urban Atlas with a majority of signatures being linked to a few
of Urban Atlas classes. Within relevant classes, we see a tendency of signature types to
cluster within Urban Atlas classes based on the level of urbanity, albeit not as strong
as in the previous two cases.
The main reason behind such a large difference are the aims of both classifications. While
the Copernicus Urban Atlas attemps to capture land cover, resulting in a large number
of non-urban classes, spatial signatures are aimed at urban environment with 13 out of 16
classes covering primarily urbanised areas.

\begin{figure}
    \centering
    \includegraphics[width=\linewidth]{fig/crosstab_ua.pdf}
    \caption{Contingency table showing frequencies (in \%) of Urban Atlas classes within signature types.}
    \label{fig:crosstab_ua}
\end{figure}

\subsection*{Local Climate Zones}
% - Local climate zones (Spsig)
    % description of dataset + figure
Local climate zones (LCZ) are conceptual classes originally designed to support study of urban
climate as temperature. It consists of 17 classes of which 10 can be classified as urban
and 7 and natural ones. In the context of Great Britain, the dataset used in this study
does not contain 2 of them, \textit{Lightweight low-rise} and \textit{Compact highrise}
as they are not present in the British landscape. The datasets produced by
\cite{demuzere2019mapping} released LCZs in a 100 meters grid based on the 2016 data. As
the LCZs are remotely sensed in this case, authors report overall average accuracy of 80 \%.
    % expectations regarding similarity
As a conceptual classification aimed to cover all possible types of primariliy urban climate zones globally,
LZCs may not be optimal when looking into a single country with specific history of urban
development. This is furhter indicated by classes that are missing. It is therefore likely
that large parts of British cities will fall into only a few of LCZ classes, while being representated
by a much larger number of signature types.
    % results + contingency table figure

Signature typology is rasterized and linked to the LCZ grid.
The resulting contingency table is shown in Figure \ref{fig:crosstab_lcz}. There is a
significant relationship between two typologies, $\chi^{2} (225, N = 16203338) = 18467242,
p < .001$. The strength of association measured as Cramér's $V$ is $0.276$, indicating
a modest to weak association, close to values we've seen in first two cases. As expected,
urban signature types are clustered primarily within \textit{Compact midrise} and
\textit{Open lowrise} LCZs, while non-urban signatures mostly fall into the \textit{Low plants} LCZ.

The difference between signatures and LCZs can be accounted to two aspects. One, as we've seen
before is the inclusion of function in spatial signatures, differentiating e.g. LCZ's \textit{Open lowrise} into
many signature types. The other is data-driven nature of signatures compared to conceptual LCZs,
where differences in signature types are below the resolution capability of simple matrix composed of
density and compactness levels. On the other, it is encouraging to see that most of signature types
fall predominantly in a single LCZ class, suggesting that while both classifications are built differently,
they are able to capture similar large-scale patterns in cities.

\begin{figure}
    \centering
    \includegraphics[width=.8\linewidth]{fig/crosstab_lcz.pdf}
    \caption{Contingency table showing frequencies (in \%) of Local Climate Zones within signature types.}
    \label{fig:crosstab_lcz}
\end{figure}

\subsection*{Summary}
None of the comparisons shows more than a moderate association, but since none of the
comparison datasets is aiming to capture the same conceptualization of space as spatial
signatures do, such a result is expected. The moderate association with both WorldPop
settlements patterns and MODUM is reassuring as both are conceptually closer to
signatures than the Urban Atlas (especially in their unsupervised design). Urban Atlas,
though very different in its aims and methods, still shows a measurable association,
which we interpret as sign that the key structural aspects forming cities are captured by both. The
comparison exercise suggests that general patterns forming cities are shared among
signatures and existing typologies. Signature types tend to form groups when we look at
their relation to comparison classes and it is not uncommon that a single signature type
is present in multiple groups linked to different classes. However, all these groups
tend to be formed based on the similarity and illustrate the granularity of the
presented classification compared to existing datasets, allowing us to distinguish, for
example, five types of signature types forming town an city centres.




\section*{Usage Notes}

The released dataset is following the widespread standards for geographic data storage and
should not pose a challenge for researchers wanting to reuse it. However, due to the density of
signature geometry (resulting from the detailed ETCs), it may be needed to simplify the
geometry for a smoother interactive experience on machines with limited resources.

Replication of the analysis optimally requires at least a single computational node with
a large amount of RAM (100 GB+) due to the size of the input data and detail on which
signature characterization is computed. It is also recommended to revisit the state of
the development of related software packages, notably momepy, libpysal, tobler and
dask-geopandas as they may soon offer more efficient drop-in replacements
of the custom code used to produce this dataset.

\section*{Code availability}

The source code used to produce this dataset is openly available in a GitHub repository
at
\hyperlink{https://github.com/urbangrammarai/spatial\_signatures}{https://github.com/urbangrammarai/spatial\_signatures}
and in the form of a website on
\hyperlink{https://urbangrammarai.github.io}{https://urbangrammarai.github.io}.
Code is
organized in a series of Jupyter notebooks and have been executed within the darribas:gds\_env
Docker container, unless specified otherwise in the individual notebooks.

\bibliography{refs}

% \noindent LaTeX formats citations and references automatically using the bibliography
% records in your .bib file, which you can edit via the project menu. Use the cite command
% for an inline citation, e.g. \cite{Kaufman2020, Figueredo:2009dg, Babichev2002,
% behringer2014manipulating}. For data citations of datasets uploaded to e.g.
% \emph{figshare}, please use the \verb|howpublished| option in the bib entry to specify
% the platform and the link, as in the \verb|Hao:gidmaps:2014| example in the sample
% bibliography file. For journal articles, DOIs should be included for works in press that
% do not yet have volume or page numbers. For other journal articles, DOIs should be
% included uniformly for all articles or not at all. We recommend that you encode all DOIs
% in your bibtex database as full URLs, e.g. https://doi.org/10.1007/s12110-009-9068-2.

\section*{Acknowledgements} (not compulsory)

% Acknowledgements should be brief, and should not include thanks to anonymous referees
% and editors, or effusive comments. Grant or contribution numbers may be acknowledged.

xxxTODOxxx: Dani do you know how to acknowledge the ESRC grant?

\section*{Author contributions statement}

% Must include all authors, identified by initials, for example: A.A. conceived the
% experiment(s), A.A. and B.A. conducted the experiment(s), C.A. and D.A. analysed the
% results. All authors reviewed the manuscript.

M.F. and D.A. designed the method, M.F. conducted the experiments, M.F. and D.A.
analysed the results. M.F. and D.A. written and reviewed the manuscript.

\section*{Competing interests}

The authors declare no competing interests.

\section*{Figures \& Tables}

xxxTODOxxx move figures and tables here

% Figures, tables, and their legends, should be included at the end of the document.
% Figures and tables can be referenced in \LaTeX{} using the ref command, e.g. Figure
% \ref{fig:stream} and Table \ref{tab:example}.

% Authors are encouraged to provide one or more tables that provide basic information on
% the main ‘inputs’ to the study (e.g. samples, participants, or information sources) and
% the main data outputs of the study. Tables in the manuscript should generally not be
% used to present primary data (i.e. measurements). Tables containing primary data should
% be submitted to an appropriate data repository.

% Tables may be provided within the \LaTeX{} document or as separate files (tab-delimited
% text or Excel files). Legends, where needed, should be included here. Generally, a Data
% Descriptor should have fewer than ten Tables, but more may be allowed when needed.
% Tables may be of any size, but only Tables which fit onto a single printed page will be
% included in the PDF version of the article (up to a maximum of three).

% Due to typesetting constraints, tables that do not fit onto a single A4 page cannot be
% included in the PDF version of the article and will be made available in the online
% version only. Any such tables must be labelled in the text as ‘Online-only’ tables and
% numbered separately from the main table list e.g. ‘Table 1, Table 2, Online-only Table
% 1’ etc.

% \begin{figure}[ht]
% \centering
% \includegraphics[width=\linewidth]{stream}
% \caption{Legend (350 words max). Example legend text.}
% \label{fig:stream}
% \end{figure}


% \begin{table}[ht]
% \centering
% \begin{tabular}{|l|l|l|}
% \hline
% Condition & n & p \\
% \hline
% A & 5 & 0.1 \\
% \hline
% B & 10 & 0.01 \\
% \hline
% \end{tabular}
% \caption{\label{tab:example}Legend (350 words max). Example legend text.}
% \end{table}

\end{document}
