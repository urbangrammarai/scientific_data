\documentclass[fleqn,10pt]{wlscirep}
\usepackage[utf8]{inputenc}
\usepackage[T1]{fontenc}
\usepackage{lineno}
\linenumbers

\title{Geographical Characterisation of British Urban Form and Function using
the Spatial Signatures Framework}

\author[1, *]{Martin Fleischmann}
\author[1]{Daniel Arribas-Bel}
\affil[1]{Geographic Data Science Lab, Department of Geography and Planning, University
of Liverpool, Roxby Building , 74 Bedford St S , Liverpool , L69 7ZT, United Kingdom}

\affil[*]{corresponding author(s): Martin Fleischmann (m.fleischmann@liverpool.ac.uk)}


\begin{abstract}
% This is a manuscript template for Data Descriptor submissions to \emph{Scientific Data}
% (\href{http://www.nature.com/scientificdata}{http://www.nature.com/scientificdata}). The
% abstract must be no longer than 170 words, and should succinctly describe the study, the
% assay(s) performed, the resulting data, and the reuse potential, but should not make any
% claims regarding new scientific findings. No references are allowed in this section.

% intro
The spatial arrangement of the building blocks that make up cities matters  to
understand the rules directing their dynamics.
% one line summary
Our study outlines the development of the national open-source classification of space
according to its form and function into a single typology.
% input
We create a bespoke granular spatial unit, the enclosed tessellation, and measure
characters capturing its form and function within a relevant spatial context.
% method
Using K-Means clustering of individual enclosed tessellation cells, we generate
a classification of space for the whole of Great Britain.
% result
Contiguous enclosed tessellation cells belonging to the same class are merged
forming spatial signature geometries and their typology.
% findings
We identify 16 distinct types of spatial signatures stretching from wild countryside,
through various kinds of suburbia to types denoting urban centres according to their
regional importance.
% use case
The open data product presented here has the potential to serve as boundary delineation for other
researchers interested in urban environments and policymakers looking for a unique
perspective on cities and their structure.

\end{abstract}
\begin{document}

\flushbottom
\maketitle
%  Click the title above to edit the author information and abstract

\thispagestyle{empty}

% \noindent Please note: Abbreviations should be introduced at the first mention in the
% main text – no abbreviations lists or tables should be included. Structure of the main
% text is provided below.

\section*{Background \& Summary}

%       (700 words maximum) An overview of the study design, the assay(s) performed, and the
%       created data, including any background information needed to put this study in the
%       context of previous work and the literature. The section should also briefly outline the
%       broader goals that motivated the creation of this dataset and the potential reuse value.
%       We also encourage authors to include a figure that provides a schematic overview of the
%       study and assay(s) design. The Background \& Summary should not include subheadings.
%       This section and the other main body sections of the manuscript should include citations
%       to the literature as needed.

% This section should provide an overview of the study that generated the data, as well as outlining the potential reuse value of the data. Any previous publications that used these data, in whole or in part, should be cited and briefly summarized.

% + Form vs function in classification & missing link
% + detailed x scalable x consistent
% + *largely rephrase main arguments from conceptual paper*
% + summary - spsig in the whole GB
%     - classification and description of space


% Form & Function in cities and their value
How the building blocks that make up cities are spatially arranged is worth
quantifying and understanding.
%% What FF is
By "building blocks", we mean both the activities and agents that inhabit
cities, as well as the (infra)structure that supports them. The
former can be conceptualised as \textit{urban function}, while the latter
falls under the study of \textit{urban form}.
%% Why:
Understanding urban form and function is important for two main reasons.
%%% encodes
First, the combination of both \textit{encodes} rich information about the
history, character and evolution of cities.
%
For example, the shape and properties of the street network encode the technology of the
time (e.g., automobile); while the degree of mix in land uses can reflect
cultural values.
%%% influences
Second, the spatial pattern of urban form and function also acts as a
frame that \textit{influences} a variety of outcomes, from economic
productivity to socio-economic cohesion to environmental sustainability.

% We use the Spatial Signatures
In this paper, we use the Spatial Signatures framework \cite{dab_mf_2021a, dab_mf_2021b},
which develops a ``characterisation of space based on form and function
designed to understand urban environments''\cite{dab_mf_2021a}.
%% Signature definition
Spatial signatures are theory-informed, data-driven computable classes that
describe the form and function of a consistent patch of geography.
%
Figure \ref{fig:workflow} presents an overview of the development of a spatial
signature classification.
%
We build a series of enclosures that we combine with building footprints
to further subdivide geographical space into what we call enclosed tessellation cells (ETCs). We then attach form and function
characters to each of these subdivisions, and use those to group them into
consistent and differentiated classes we call signatures.
%
Each phase is expanded in detail in the next section.

    % In this paper
% Great British Signatures
We introduce an open data product (ODP \cite{odp_paper21}) containing a classification of
spatial signatures for Great Britain (illustrated in a figure \ref{fig:map}). In doing so, we provide an
analysis-ready layer that brings together urban form and function consistently, in
detail, and at national scale. To the best of our knowledge, this is the first
dataset capturing urban form and function published both with a degree of detail and scale as
ours.
%% Input data used
Our results are based on the analysis of more than 14 million of ETCs, to
each of which we attach more than 300 characters capturing a wide range of
aspects relating to urban form and function.
%% Data available
We provide access to both granular geographical boundaries of the delineated spatial signatures
as well as measurements for each character at the signature level.
%% Web map and code
The ODP also includes a web map that allows exploration without any technical
requirement other than a web browser, and we have open sourced all the code,
including details on the computational backend.
%% Comparison with other datasets
The uniqueness of our ODP makes it challenging to set up a technical validation as a
comparison with existing datasets. Nevertheless, we relate our
signatures to a few well-established data products that capture each a subset
of the form and function dimensions we consider. Our results are encouraging
in that they show broad agreement in expected areas, but also highlight
aspects that can only be discovered when considering form and function in tandem.

% Benefits of signature approach (goals why we created it)
The approach and outputs presented bring several benefits to a range of
stakeholders interested in cities.
%% ODP
This spatial signatures ODP provides insight generated from detailed,
comprehensive and computationally intensive data analysis and presents it in a
way that is easy to access, work with and integrate into larger projects.
%
%%% Academics and policymakers
Together with the importance of form and function discussed above, we
anticipate the output will be relevant to both academic researchers as well as
policymakers and practitioners.
%% Broader SS benefits
As a conceptual framework, the spatial signatures provide a flexible yet
generalisable approach to understand, characterise and quantify urban form and
function. One way to understand our results is as an implementation of a
more general way of thinking about the spatial dimension of cities. In this
context, it can be useful to researchers and practitioners who, even if not
specifically interested in Great Britain, would like to implement a similar
approach.
%
In this respect, we hope the present paper serves not only to document our own
work but to inspire future efforts aimed at urban form and function.



            %%% FROM ORIGNAL GUIDELINE %%%

% An overview of the study design, the assay(s) performed, and the created data, including any background information needed to put this study in the context of previous work and the literature.

%- Briefly outline the broader goals that motivated the creation of this dataset and the potential reuse value.

%- We also encourage authors to include a figure that provides a schematic overview of the study and assay(s) design.




\section*{Methods}

% conceptualisation of signature detection
% small unit
% complex descirption (F&F)
% cluster analysis
The method of identification of spatial signatures consists of three top-level steps.
First, we delineate a spatial unit of analysis that reflects the structure of
urban phenomena on a very granular level. Then we characterise each of them
according to form and function, capturing the nature of each unit and its spatial
context. Finally, we use cluster analysis to derive a typology of our spatial units
that, once combined into contiguous areas, forms a typology of spatial signatures.

\subsection*{Spatial unit}
% spatial unit
    % conceptualisation of enclosed tessellation
    % rules
        % indivisible
        % internally consistent
        % geographically exhaustive
    % options
        % admin boundaries
        % arbitrary grids
        % morphological units
    % ET design
        % barriers
        % enclosures
        % anchors
        % ET cells
The first major methodological decision relates to the definition of the
spatial unit. An ideal candidate needs to reflect space in a granular manner, and we argue
it should fulfil three conditions. First, it should be \textit{indivisible},
meaning that any subdivision would result in a unit that is incapable of
capturing the nature of urban form and function. Second, it needs to be
\textit{internally consistent} - it should always reflect only a single signature type.
Last, it should be geographically \textit{exhaustive}, covering the entirety of the study
area.

Spatial units used in literature can be split into three groups. One is using
administrative boundaries like city regions\cite{angel2020}, wards or census output areas\cite{alexiou2016}, that are
convenient to obtain and can be easily linked to auxiliary data. However,
those rarely reflect the morphological composition of urban space and, in some cases, may
even “obscure morphologic reality”\cite{taubenbock2019new}. At the same time, most of them
are divisible, and larger units are not always internally consistent. Another group is based on
arbitrary uniform grids linked either to spatial indexing methods like
H3\cite{brodsky2018h3} or Ordnance Survey
National Grid, or to ancillary data of remote sensing or other origins like a
WorldPop grid\cite{jochem2021tools}. Grids however cannot be considered internally
consistent as
they do not consider the underlying structure of the landscape.
Finally, urban morphology studies tend to use morphological elements as
street segments\cite{araldi2019}, blocks\cite{gil2012},
buildings\cite{hamaina2012a} or plots\cite{bobkova2019} as units of analysis.
Some of those
could be seen as indivisible and internally consistent, but since they are largely based
on built-up fabric, they are not exhaustive. For example, in areas without any building or
street, there
is no spatial unit to work with. Plots could be theoretically considered as exhaustive,
consistent and indivisible, but there is no accepted conceptual definition and unified
geometric representation\cite{kropf2018}.

We are, therefore, proposing an application of an alternative spatial unit called \textit{enclosed
tessellation cell} (ETC), defined as "the portion of space that results
from growing a morphological tessellation within an enclosure delineated by a series
of natural or built barriers identified from the literature on urban form, function and
perception"\cite{dab_mf_2021a}.
% We should drop a reference to the conceptual paper here.
ETCs follow the morphological tradition in that it is
based on the physical elements of an environment but overcome the drawbacks of
conventionally used units. Its geometry is generated in the three steps illustrated in a
Figure \ref{fig:et_diagram}. First, a set of features representing physical barriers
subdividing space, in our case composed of the street network, railways, rivers and a
coastline, is combined, generating a layer of boundaries (\ref{fig:et_diagram} A).
These then partition space
into smaller enclosed geometries called \textit{enclosures} (\ref{fig:et_diagram} B),
which can be very granular
or very coarse depending on the geographic context. In dense city centres where a single
enclosure represents a single block is a high frequency of small enclosures. At the same time, in the
countryside, this approach leads to very few large enclosures as their delimiters are far away
from each other. Enclosures are then combined with building footprints (\ref{fig:et_diagram} B),
which act as
anchors in space and potentially subdivide enclosures into enclosed tessellation cells using the
morphological tessellation algorithm\cite{fleischmann2020} (\ref{fig:et_diagram} D),
a polygon-based adaptation of Voronoi
tessellation. The resulting geometries are indivisible as they contain, at most, a single
anchor building, internally consistent due to their granularity and link to morphological
elements composing urban fabric, and geographically exhaustive as they cover an entire area
limited by specified boundaries.

    % data input
        % barries
            % roads from OS Open Roads
                % data description (simplified centerlines)
            % railway from OS OpenMap - Local
                % data description (simplified centerlines)
            % rivers from OS OpenRivers
                % data description (simplified centerlines)
            % coastline from OS Strategi®
                % data description (continuous enclosing geometry)
        % buildings
            % OS OpenMap - Local
            % data quality description (merged adjacent geometries)

In our ODP for Great Britain, street networks are extracted from OS
Open Roads datasets\cite{openroads2020} representing simplified road centrelines
cleaned of underground road segments.
Railways are retrieved from OS OpenMap - Local\cite{openmap2020}
("RailwayTrack" layer) which captures surface railway tracks. Rivers are extracted from
OS OpenRivers\cite{openrivers2020} representing river network of GB as centrelines, and a coastline is
retrieved from OS Strategi®\cite{strategi2016}, capturing coastline as a continuous line
geometry. Building geometry is extracted, again, from OS OpenMap - Local ("Building"
layer) and represents generalised building footprint polygons.\footnote{Note that the dataset
does not distinguish between individual buildings when they are adjacent (e.g. perimeter
block composed of multiple buildings is represented by a single polygon).}

% characterisation of space
\subsection*{Characterisation of space}
Spatial signatures capture the character of the built and unbuilt environment
based on two components - form and function. Each of them is quantified at the level of
individual ETCs using methods appropriate for each specific dataset. While form
is described using urban morphometrics (i.e. quantitative analysis of urban
form)\cite{dibble2019origin}, function is a composite of a variety of data inputs. We outline each
component with a bit more detail below.

\subsubsection*{Form}
    % form
        % input data
            % ET cells
            % bulidings
            % street network
        % morphometrics
            % different categories of characters
                % dimension
                % shape
                % spatial distribution
                % intensity
                % connectivity
                % diversity
            % different scales
                % individual elements -> adjacency -> networks
        % contextualisation
            % interest in characterisation of spatial patterns
            % distance-weighted higher order contiguity spatial weights
            % reflection of a statistical distribution of data within context
                % proxy of diversity
Morphometric characterisation of urban form is based on the numerical description
of four elements capturing the built environment - buildings, streets, ETCs, and
enclosures - and reflects their patterns based on six categories of
characters: dimensions, shapes, spatial
distribution, intensity, connectivity and diversity\cite{fleischmann2020a}. Each element is considered across
different scales, from the measurement of individual geometries, to relations of
neighbouring geometries, to a graph-based analysis of the street network. The combination of
elements, categories and scales results in a set of 59 individual morphometric
characters listed in the online table 1.

% Q: do we want to include some morphometric theory here? I'd say no since this is a
% data descriptor but I can add some if needed.

However, measuring individual characters is not enough to understand the predominant
spatial patterns. For some types of urban environment, high heterogeneity is not
uncommon. This means that using, for example, areas of building footprints would, in most cases, result
in largely discontinuous clusters that do not capture the pattern within an area. Therefore, we represent each of the
morphometric characters using three summary variables reflecting statistical distributions
of measured data within a spatial context of each ETC. Context is defined as
tenth
order of contiguity computed across the mesh composed of contiguous ETCs. Furthermore, each
value is weighted by the inverse distance between so-called poles of inaccessibility
(defined as a centre of a maximum inscribed circle) of each ETC. Three proxy variables
then capture the first, the second and the third quartile of the resulting weighted
distribution. Such a characterisation can capture the contextual tendency of each
morphometric character and hence identify contiguous clusters in both homogenous and
heterogeneous urban tissues.

\subsubsection*{Function}
    % function
        % input data
            % overview - from population and POIs to NDVI and night lights
            % include table with transfer methods
        % transfer methods overview
            % blg-based interpolation
            % areal interpolation
            % accessibility
            % euclidean distance
            % zonal statistics
        % contextualisation
            % depending on the transfer method, function-based characters we
            %  contextualised using the same method used in morphometrics
Characterisation of the function component uses a different approach. While data
describing urban form are not generally available in a processed format, forcing us to employ morphometric approaches, different aspects of function are often available as
open data products. Therefore, the main goal of our characterisation of ETCs based on
function is to develop appropriate transfer methods to link data published as grids or
linked to administrative boundaries to ETCs.

In this work, we are using five different transfer methods: Areal interpolation,
Building-based dasymetric areal interpolation\cite{eli_knaap_2021_5047613}, Network-constrained accessibility,
Euclidean accessibility, and Zonal statistics. Areal interpolation is used when the
functional data covers the entirety of space in the
form of polygon geometry and when there is no assumption that the phenomena it captures
are linked directly to the human population, such as land cover data. When there is
an assumption of relation to the population, building-based dasymetric areal
interpolation is used instead. The main difference is that instead of ETC polygons,
building footprint polygons linked to individual ETCs are used as a target of
interpolation. That ensures that data like population estimates are linked to ETCs
proportionally to their ability to house population rather than by their area.
Network-constrained accessibility is used when the input data represents points of
interest like locations of supermarkets. Points are then snapped to the nearest node on
the street network and linked to the ETCs through the count of observations
accessible from the cell within 15 minutes of walk (1200m on the street network) and a distance to the nearest point. In
some cases, Euclidean (as-crow-flies) accessibility is measured instead to accommodate
for phenomena that are often outside the reach of a drivable network like water bodies.
Zonal statistics are used to transfer data originally stored in a raster
format to ETCs as the mean value of raster pixels intersecting each polygon
geometry. Finally, characters based on interpolation and zonal statistics are expressed
using their contextual versions following the method used for form characters to, again,
reflect the contextual pattern of measured values. The selection of datasets and the chosen
transfer method are listed in the table \ref{tab:function}.

\subsection*{Cluster analysis}
% cluster analysis
    % comparison of clustering methods (shall we include this?? skip for now, keep only
    % as a notebook online)
        % K-Means, GMM, SOM

    % two levels of K-Means clustering
        % data input
            % combination of both form and function, all characters equally weighted
            % only contextualised representation of characters is used to capture pattern

            % data standardisation
                % column-based mean standardisation

When combined, contextual summaries of form and function characters (or characters
themselves when they are reflecting the context by definition) compose a dataset
describing each ETC by 331 variables (177 for form and 154 for function.)
Assigning equal weight to each variable, we standardize them applying
Z-score normalization, and use them as input for K-Means cluster analysis.

        % selection of number of clusters
            % clustergram
            % supplementary metrics
                % Silhouette
                % Calinski-Harabasz
                % Davies-Boulding
Due to the nature of the selected K-Means clustering, the step preceding the final
analysis is the selection of an optimal number of clusters. We use the
clustergram exploratory method\cite{schonlau2002clustergram}, reflecting the behaviour of different options, the relationship
between clustering solutions regarding the allocation of individual observations to
classes, and the separation between the clusters within each tested solution.
Clustergram is further accompanied by measures of internal validation measures - the
Silhouette score diagram, Calinski-Harabasz index\cite{calinski1974} and Davies-Bouldin index\cite{davies1979cluster}. The optimal
number of classes is selected based on the interpretation of clustergram supported by
additional measures aiming at a balance between cluster separation and an appropriate
detail of resulting classification.

        % top level providing a first national classification
        % sub clustering of urban areas
            % selection criteria for to-be-subclustered classes
The results of the clustering capture the first group of a national signature
classification composed of ten clusters. However, since the classified ETCs cover entirety of space from vast
natural open spaces to dense city centres, it may result in only a few classes
representing urban areas. While that is caused by the variable heterogeneity of our
dataset in combination with K-Means clustering, the measured characters have the ability
to further distinguish classes of already identified clusters. As spatial signatures
are focused on the urban environment, we further subdivide those clusters covering
a substantial portion of urban areas using another iteration of K-Means clustering
(one class into nine and another into three clusters).
The resulting classification then provide classification capturing the typology of
spatial signatures with a detailed focus on urban development.


% generation of signature geometry
    % dissolution based on contiguity and assigned class
Finally, individual spatial signature geometries are generated as a combination of
adjacent ETCs belonging to the same signature class.

% - Feature importance - should this actually be included in here? It is not part of the
%   data product so I'd say no.


\section*{Data Records}

% The Data Records section should be used to explain each data record associated with this
% work, including the repository where this information is stored, and to provide an
% overview of the data files and their formats. Each external data record should be cited
% numerically in the text of this section, for example \cite{Hao:gidmaps:2014}, and
% included in the main reference list as described below. A data citation should also be
% placed in the subsection of the Methods containing the data-collection or analytical
% procedure(s) used to derive the corresponding record. Providing a direct link to the
% dataset may also be helpful to readers
% (\hyperlink{https://doi.org/10.6084/m9.figshare.853801}{https://doi.org/10.6084/m9.figshare.853801}).

% Tables should be used to support the data records, and should clearly indicate the
% samples and subjects (study inputs), their provenance, and the experimental
% manipulations performed on each (please see 'Tables' below). They should also specify
% the data output resulting from each data-collection or analytical step, should these
% form part of the archived record.

%  This section should be used to explain each data record associated with this work,
%  including the repository where this information is stored, and to provide an overview
%  of the data files and their formats. Each external data record should be cited using
%  our data citation format.

% description of the GPKG/archive
The data product described in this article is available through the Consumer Data
Research Centre Open Data repository available at
\hyperlink{https://data.cdrc.ac.uk/dataset/spatial-signatures-great-britain}{https://data.cdrc.ac.uk/dataset/spatial-signatures-great-britain}
under the Open Government Licence v3.0 license and archived at
\hyperlink{https://doi.org/10.6084/m9.figshare.16691575.v1}{https://doi.org/10.6084/m9.figshare.16691575.v1}.
The dataset stored in the repository contains a GeoPackage with a signature geometry
(OSGB36 / British National Grid (\texttt{EPSG:27700}) CRS) and related signature type,
plain-text pen portraits describing individual signature types, a series of CSV files
describing individual signatures and signature types, and a CSV files linking signature
types to the Output Area and Lower Super Output Area geometry. An online interactive map
of spatial signatures for the whole of Great Britain is available on the project website
(\hyperlink{https://urbangrammarai.xyz/great-britain}{https://urbangrammarai.xyz/great-britain}).


\section*{Technical Validation}

% This section presents any experiments or analyses that are needed to support the
% technical quality of the dataset. This section may be supported by figures and tables,
% as needed. This is a required section; authors must present information justifying the
% reliability of their data.

Spatial signatures are unique as a classification method, limiting the potential
validation approaches. In this context, we focus on indirect methods that use ancillary datasets capturing
conceptually similar aspects of the environment. We compare the signatures with three of
such datasets, each focusing on a different classification perspective, but all related
to our classification to a degree when we can assume there will be a measurable level of
association between the two:

\begin{itemize}
    \item WorldPop settlement patterns of building footprints (2021)\cite{jochem2021tools}
    \item Classification of Multidimensional Open Data of Urban Morphology (MODUM) (2015)\cite{alexiou2016}
    \item Copernicus Urban Atlas (2018)\cite{eea2018}
\end{itemize}


\subsection*{Validation approach}
% General method of validation
    % data transfer (one or the other way depending on feasibility) chi-squared
    % statistic Cramer's V
All datasets, spatial signatures and those selected as validation contain a
categorical classification of space linked to their unique geometry. The first
requirement to be able to compare data products is to transfer their
information to the same geometry. We take two approaches for this step,
depending on the dataset we are comparing the signatures with:
an interpolation of one set of polygon-based data to another (input to ETCs);
or the conversion of
spatial signatures to the raster representation matching an input raster,
which is computationally more efficient when one of the layers is already a raster. The second
step is a statistical comparison of two sets of classification labels, one representing
spatial signature typology and the other validation classes. We use contingency tables
and Pearson's $\chi^{2}$ test to determine whether the frequencies of observed
(signature types) and expected (validation types) labels significantly differ in one or
more categories. Furthermore, we use Cramér's $V$ statistics\cite{cramer2016mathematical} to assess the strength of
the association.

\subsection*{WorldPop settlement patterns of building footprints}
% - WorldPop (Spsig)
    % description of dataset + figure
WorldPop settlement patterns of building footprints dataset aims to derive a typology of
morphological patterns based on a gridded approach with cells of
100x100m, and building footprints. Authors measure six morphometric characters
linked to the grid cells and use them as input for an unsupervised clustering
algorithm leading to a six-class typology.
    % expectations regarding similarity
As the classification is dependent on building footprints, grid cells that do
not contain any information on the building-based pattern are treated as missing in the
final data product. For the validation of spatial signatures, this \textit{missing}
category is treated as a single class. It is assumed that the top-level large scale
patterns detected by the WorldPop method and spatial signatures will provide similar
results. However, there will be differences caused by the inclusion of function in spatial
signatures, higher granularity of both initial spatial units and the resulting
classification (6 vs 19 classes).

Signature typology is rasterized and linked to the WorldPop grid. The resulting
contingency table is shown in Figure \ref{fig:crosstab_worldpop}. There is a significant relationship between
two typologies, $\chi^{2} (114, N = 22993921) = 13341832, p < .001$. The strength of
association measured as Cram\'{e}r's $V$ is $0.311$, indicating moderate association.
The contingency table shows that WorldPop classes tend to be linked to groups of
signature types of a similarly degree of urbanity. A WorldPop class 15 is "undefined" due
to the lack of building footprints in the area, therefore overlapping a large portion of
signatures.
    % results + contingency table figure

\subsection*{MODUM}
% - MODUM (Spsig)
    % description of dataset + figure
Multidimensional Open Data Urban Morphology (MODUM) classification describes a typology
of neighbourhoods derived from 18 indicators capturing built environment as streets,
railways or parks, linked to the Census Output Area geometry. The classification
identifies 8 types of neighbourhoods.
    % expectations regarding similarity
Compared to the WorldPop classification, MODUM takes into account more features of
the built environment than building footprints, which makes it conceptually closer to the
spatial signatures. However, it is still focusing predominantly on the form component,
although there are some indicators that would be classified as function within the
signatures framework (e.g. population). The MODUM method uses a different way of
capturing context compared to the signatures, which leads to some classes being
determined predominantly by a single character. For example, the \textit{Railway Buzz} type
forms a narrow strip around the railway network, which is an effect signatures avoid.
    % results + contingency table figure
MODUM typology is available only for England and Wales. Therefore the validation takes
into account only ETCs covering the same area. The classification is linked to the
ETC geometry is based on the proportion (the type covering the largest portion of ETC is
assigned). The resulting contingency table is shown in Figure \ref{fig:crosstab_modum}. There is a
significant relationship between two typologies, $\chi^{2} (152, N = 13067584) =
13938867, p < .001$. The strength of association measured as Cramér's $V$ is $0.300$,
indicating moderate association of very similar levels we have seen above. The
contingency table indicates similar relationships, where a single MODUM class overlaps a
group of signature types. However, the groups tend to be well defined and formed based
on the similarity of types.

\subsection*{Copernicus Urban Atlas}
% - Urban atlas (Spsig)
    % description of dataset + figure
Copernicus Urban Atlas is the least similar of the validation datasets. It is a
high-resolution land use classification of functional urban areas derived primarily from
Earth Observation data enriched by other reference data as OpenStreetMap or topographic
maps. Its smallest spatial unit in urban areas is 0.25 ha and 1 ha in rural areas,
defined primarily by physical barriers. It identifies
27 predefined classes using the supervised method.
    % expectations regarding similarity
The majority of urban areas is classified as urban fabric further distinguished based on
continuity and density resulting in six classes of the urban fabric. The classification does
not consider the type of the pattern or any other aspect. Furthermore, it does not take
into account what signatures call \textit{context} as each spatial unit is
classified independently, which in some cases leads to the high heterogeneity of
classification within a small portion of land. Signatures take a different approach.
Consequently, it is expected that the similarity between the two will be limited.
    % results + contingency table figure
Urban Atlas is available only for functional urban areas (FUA), leaving rural areas
unclassified. Validation then applies to FUAs only. The classification is linked to the
ETC geometry based on the proportion (the type covering the largest portion of ETC is
assigned). The resulting contingency table is shown in Figure \ref{fig:crosstab_ua}. There is a
significant relationship between two typologies, $\chi^{2} (450, N = 8396642) = 5229900,
p < .001$. The strength of association measured as Cramér's $V$ is $0.186$, indicating
a weak association. The contingency table shows the difference in the aim of spatial
signatures and that of Urban Atlas with a majority of signatures being linked to a few
of Urban Atlas classes. Within relevant classes, we see a tendency of signature types to
cluster within Urban Atlas classes based on the level of urbanity, albeit not as strong
as in the previous two cases.

\subsection*{Summary}
None of the comparisons shows more than a moderate association, but since none of the
validation datasets is aiming to capture the same conceptualization of space as spatial
signatures do, such a result is expected. The moderate association with both WorldPop
settlements patterns and MODUM is reassuring as both are conceptually closer to
signatures than the Urban Atlas (especially in their unsupervised design). Urban Atlas,
though very different in its aims and methods, still shows a measurable association,
which we interpret as sign that the key structural aspects forming cities are captured by both. The
validation exercise suggests that general patterns forming cities are shared among
signatures and existing typologies. Signature types tend to form groups when we look at
their relation to validation classes and it is not uncommon that a single signature type
is present in multiple groups linked to different classes. However, all these groups
tend to be formed based on the similarity and illustrate the granularity of the
presented classification compared to existing datasets, allowing us to distinguish, for
example, five types of signature types forming town an city centres.


\section*{Usage Notes}

The released data product follows widespread standards for geographic data storage and
should be easy to integrate with other data and methods by researchers wanting to reuse it. However, due to the density of
signature geometry (resulting from the detailed ETCs), it may be needed to simplify the
geometry for a smoother interactive experience on machines with limited resources.

Replication of the analysis optimally requires at least a single computational node with
a large amount of RAM (+100GB) due to the size of the input data and detail on which
signature characterization is computed. It is also recommended to revisit the state of
the development of related software packages, notably
\texttt{momepy}\cite{fleischmann_2019}, \texttt{libpysal}\cite{rey2021pysal},
\texttt{tobler}\cite{eli_knaap_2021_5047613} and
\texttt{dask-geopandas} as they may soon offer more efficient drop-in replacements
of the custom code used to produce this dataset.

\section*{Code availability}

The source code used to produce this dataset is openly available in a GitHub repository
at \newline
\hyperlink{https://github.com/urbangrammarai/spatial\_signatures}{https://github.com/urbangrammarai/spatial\_signatures}
and in the form of a website on
\hyperlink{https://urbangrammarai.xyz}{https://urbangrammarai.xyz}.
Code is
organized in a series of Jupyter notebooks and have been executed within the
\texttt{darribas:gds\_dev:6.1}\cite{gds_env}
Docker container, unless specified otherwise in the individual notebooks.

\bibliography{refs}

% \noindent LaTeX formats citations and references automatically using the bibliography
% records in your .bib file, which you can edit via the project menu. Use the cite command
% for an inline citation, e.g. \cite{Kaufman2020, Figueredo:2009dg, Babichev2002,
% behringer2014manipulating}. For data citations of datasets uploaded to e.g.
% \emph{figshare}, please use the \verb|howpublished| option in the bib entry to specify
% the platform and the link, as in the \verb|Hao:gidmaps:2014| example in the sample
% bibliography file. For journal articles, DOIs should be included for works in press that
% do not yet have volume or page numbers. For other journal articles, DOIs should be
% included uniformly for all articles or not at all. We recommend that you encode all DOIs
% in your bibtex database as full URLs, e.g. https://doi.org/10.1007/s12110-009-9068-2.

\section*{Acknowledgements} (not compulsory)

% Acknowledgements should be brief, and should not include thanks to anonymous referees
% and editors, or effusive comments. Grant or contribution numbers may be acknowledged.

M.F. and D.A. kindly acknowledge funding by the UK's Economic and Social Research
Council through the project ``Learning an urban grammar from satellite data
through AI'', project reference \texttt{ES/T005238/1}.

\section*{Author contributions statement}

% Must include all authors, identified by initials, for example: A.A. conceived the
% experiment(s), A.A. and B.A. conducted the experiment(s), C.A. and D.A. analysed the
% results. All authors reviewed the manuscript.

M.F. and D.A. designed the method, M.F. conducted the experiments, M.F. and D.A.
analysed the results. M.F. and D.A. wrote and reviewed the manuscript.

\section*{Competing interests}

The authors declare no competing interests.

\section*{Figures \& Tables}


\begin{figure}[ht]
    \includegraphics[width=\linewidth]{fig/workflow.pdf}
\caption{Diagram illustrating the sequential steps leading to the delineation of
spatial signatures. From a series of enclosing components, to enclosures,
enclosed tessellation (ET), the addition of form and function characters to ET
cells, and the development of spatial signatures.}
\label{fig:workflow}
\end{figure}


\begin{figure}[ht]
    \includegraphics[width=\linewidth]{fig/signatures_map.png}
\caption{Illustration of a classification of spatial signatures in Liverpool and
    Birkenhead area, in the north west of England.}
\label{fig:map}
\end{figure}

\begin{figure}[ht]
    \includegraphics[width=\linewidth]{fig/et_diagram.pdf}
    \caption{Diagram illustrating the sequential steps leading to the delineation of
    enclosed tessellation. From a series of enclosing components, where blue are streets
    and yellow river banks (A), to enclosures (B), incorporation of buildings as anchors
    (C) to final tessellation cells (D).}
    \label{fig:et_diagram}
\end{figure}

\begin{figure}[ht]
    \centering
    \includegraphics[width=.8\linewidth]{fig/crosstab_worldpop.pdf}
    \caption{Contingency table showing frequencies (in \%) of WorldPop classes within signature types.}
    \label{fig:crosstab_worldpop}
\end{figure}

\begin{figure}[ht]
    \centering
    \includegraphics[width=.8\linewidth]{fig/crosstab_modum.pdf}
    \caption{Contingency table showing frequencies (in \%) of MODUM classes within signature types.}
    \label{fig:crosstab_modum}
\end{figure}

\begin{figure}[ht]
    \centering
    \includegraphics[width=\linewidth]{fig/crosstab_ua.pdf}
    \caption{Contingency table showing frequencies (in \%) of Urban Atlas classes within signature types.}
    \label{fig:crosstab_ua}
\end{figure}


\begin{table}[ht]
\begin{tabular}{p{35mm}p{50mm}p{35mm}p{40mm}}
\toprule
                                data & source &                   input geometry &                               transfer method \\
\midrule
                Population estimates & ONS Census Output Area population estimates, Statistics.gov.scot &     Vector (output area polygon) & Building-based dasymetric areal interpolation \\
            Retail POIs (supermarkets) & Geolytix &                   Vector (point) &             Network-constrained accessibility \\
                        Water bodies & OS OpenMap Local &      Vector (water body polygon) &                       Euclidean accessibility \\
                    Listed Buildings & Historic England, Historic Environment Scotland, Lle Geo-Portal for Wales &                   Vector (point) & Network-constrained accessibility \\
                        Night Lights & VIIRS DNB Nighttime Lights &                    Raster (500m) &                              Zonal statistics \\
    Food Hygiene Rating Scheme Ratings & CDRC.ac.uk &                   Vector (point) &             Network-constrained accessibility \\
                Workplace population & ONS Census Workplace population, Scotland's census Workplace population &     Vector (output area polygon) & Building-based dasymetric areal interpolation \\
            Culture (theatres, cinemas) & OpenStreetMap &                   Vector (point) &             Network-constrained accessibility \\
                    Corine land cover & Copernicus Land Monitoring Service & Vector (land cover zone polygon) &                           Areal interpolation \\
                                NDVI & GHS-composite-S2 R2020A &                     Raster (10m) &                              Zonal statistics \\
                        Retail centres & CDRC.ac.uk &   Vector (retail centre polygon) &                       Euclidean accessibility \\
\bottomrule
\end{tabular}

\caption{\label{tab:function}Functional characters used to
describe the function component of spatial signatures. For details of the implementation,
refer to the reproducible Jupyter notebooks available at urbangrammarai.xyz.}
\end{table}

% Figures, tables, and their legends, should be included at the end of the document.
% Figures and tables can be referenced in \LaTeX{} using the ref command, e.g. Figure
% \ref{fig:stream} and Table \ref{tab:example}.

% Authors are encouraged to provide one or more tables that provide basic information on
% the main ‘inputs’ to the study (e.g. samples, participants, or information sources) and
% the main data outputs of the study. Tables in the manuscript should generally not be
% used to present primary data (i.e. measurements). Tables containing primary data should
% be submitted to an appropriate data repository.

% Tables may be provided within the \LaTeX{} document or as separate files (tab-delimited
% text or Excel files). Legends, where needed, should be included here. Generally, a Data
% Descriptor should have fewer than ten Tables, but more may be allowed when needed.
% Tables may be of any size, but only Tables which fit onto a single printed page will be
% included in the PDF version of the article (up to a maximum of three).

% Due to typesetting constraints, tables that do not fit onto a single A4 page cannot be
% included in the PDF version of the article and will be made available in the online
% version only. Any such tables must be labelled in the text as ‘Online-only’ tables and
% numbered separately from the main table list e.g. ‘Table 1, Table 2, Online-only Table
% 1’ etc.

% \begin{figure}[ht]
% \centering
% \includegraphics[width=\linewidth]{stream}
% \caption{Legend (350 words max). Example legend text.}
% \label{fig:stream}
% \end{figure}


% \begin{table}[ht]
% \centering
% \begin{tabular}{|l|l|l|}
% \hline
% Condition & n & p \\
% \hline
% A & 5 & 0.1 \\
% \hline
% B & 10 & 0.01 \\
% \hline
% \end{tabular}
% \caption{\label{tab:example}Legend (350 words max). Example legend text.}
% \end{table}

\end{document}
