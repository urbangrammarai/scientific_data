% conceptualisation of signature detection
    % small unit
    % complex descirption (F&F)
    % cluster analysis
The method of identification of spatial signatures consists of three top level steps.
First, we need to delineate spatial unit of analysis, one that reflects the structure of
urban phenomena on a very granular level. Then we charactercterise each of them
according to the form and function capturing the nature of each unit and its spatial
context. Finally, we use cluster analysis to derive a typology of our spatial units
that, once combined into contiguous areas, forms a typology of spatial signatures.

\subsection*{Spatial unit}
% spatial unit
    % conceptualisation of enclosed tessellation
    % rules
        % indivisible
        % internally consistent
        % geographically exhaustive
    % options
        % admin boundaries
        % arbitrary grids
        % morphological units
    % ET design
        % barriers
        % enclosures
        % anchors
        % ET cells
The first major methodological decision needs to be taken on the definition of the
spatial unit. As mentioned, it needs to reflect space in a granular manner and we argue
that it should fulfil three conditions. First, it should be \textit{indivisible},
meaning that when such a unit would be subdivided into smaller parts, none of them would
be enough to capture the nature of spatial signature. Second, it needs to be
\textit{internally consistent} - it should always reflect only a single signature type.
Last, it should be geographically \textit{exhaustive}, covering entirety of the study
area.

Spatial units used in literature can be split into three groups. One is using
administrative boundaries like city regions, wards or census output areas, that are
convenient to obatain and can be easily linked to auxiliarry data. However,
those rarely reflect the morphological composition of urban space and in some cases may
even “obscure morphologic reality” REF Taubenbock 2019. At the same time, most of them
are divisible and larger units are not alwyas internally consistent. Another group is based on
arbitrary uniform grids linked either to spatial indexing method like H3 REF or OS
National Grid REF, or to anciliarry data of remote sensing or other origins like a
WorldPop grid REF. The issue is that grids cannot be considered internally consistent as
they have no relation to the real-life spatial pattern. Finally, urban morphology tends to use morphological elements as
street segments REF, blocks REF buildings or plots as a unit of analysis. Some of those
could be seen as indivisible and internally consistent but since they are largely based
on built-up fabric, they are not exhaustive. When there is no builiding or street, there
is no spatial unit to work with. Plots could be theoretically considered as exhaustive,
consistent and indivisible but there is no accepted conceptual definition and unified
geometric representation (REF Kropf).

We are, therefore, proposing an application of an alternative spatial unit called \textit{enclosed
tessellation cell} (ETC), defined as:

\newtheorem*{theorem}{}
\begin{theorem}
    A characterisation of space based on form and function designed to understand urban
environments
\end{theorem}

% We should drop a reference to conceptual paper here.

ETC follows the morphological tradition in a sense that is it
based on the physical elements of an enviroment but overcomes the drawbacks of
conventionally used units. Its geometry is generated in three steps illustrated on a
Figure \ref{fig:et_diagram}. First, a set of features representing physical barriers
subdividing space, in our case composed of street network, railways, rivers and a
coastline, is combined together, generating a layer of boundaries. These then partition space
into smaller enclosed geometries called \textit{enclosures}, which can be very granular
or very coarse depending on the geographic context. In dense city centres where a single
enclosure represent a single block is a high frequency of small enclosures, while in the
countryside, we can observe very few large enclosures as their delimiters are far away
from each other. Enclosures are then combined with building footprints, posing as
anchors in the space and are subdivided into enclosed tessellation cells using the
morphological tessellation algorithm REF, a polygon-based adaptation of Voronoi
tessellation. Resulting geometries are indivisible as they contain, at most, a single
achor building, internally consistent due to their granularity and link to morphological
elements composing urban fabric, and geogrpahically exhasutive as they cover entire area
limited by specified boundaries.

    % data input
        % barries
            % roads from OS Open Roads
                % data description (simplified centerlines)
            % railway from OS OpenMap - Local
                % data description (simplified centerlines)
            % rivers from OS OpenRivers
                % data description (simplified centerlines)
            % coastline from OS Strategi®
                % data description (continuous enclosing geometry)
        % buildings
            % OS OpenMap - Local
            % data quality description (merged adjacent geometries)

In the case of classification of Great Britain, street networks are extracted from OS
Open Roads datasets (REF) representing simplified road centrelines cleaned of road
segments under the ground. Railways are retrieved from OS OpenMap - Local
("RailwayTrack" layer) which captures surfance railway tracks. Rivers are extracted from
OS OpenRivers (REF) representing river network of GB as centrelines, and a coastline is
retrieved from OS Strategi® (2016) REF, capturing  coastline as a continous line
geometry. Building geometry is extracted, again, from OS OpenMap - Local ("Building"
layer) and represents generalised building footprint polygons. Note that the dataset
does not distinguish between individual buildings when they are adajcent (e.g. perimeter
block composed of multiple buildings is represent by a single polygon).

% characterisation of space
    % form
        % input data
            % ET cells
            % bulidings
            % street network
        % morphometrics
            % different categories of characters
                % dimension
                % shape
                % spatial distribution
                % intensity
                % connectivity
                % diversity
            % different scales
                % individual elements -> adjacency -> networks
        % contextualisation
            % interest in characterisation of spatial patterns
            % distance-weighted higher order contiguity spatial weights
            % reflection of a statistical distribution of data within context
                % proxy of diversity

    % function
        % input data
            % overview - from population and POIs to NDVI and night lights
            % include table with transfer methods
        % transfer methods overview
            % spatial join
            % interpolation
            % accessibility
        % contextualisation
            % depending on the transfer method, function-based characters we
            %  contextualised using the same method used in morphometrics

% cluster analysis
    % comparison of clustering methods (shall we include this?? skip for now)
        % K-Means, GMM, SOM

    % two levels of K-Means clustering
        % data input
            % combination of both form and function, all characters equally weighted
            % only contextualised representation of characters is used to capture pattern

            % data standardisation
                % column-based mean standardisation

        % selection of number of clusters
            % clustergram
            % supplementary metrics
                % Silhouette
                % Calinski-Harabasz
                % Davies-Boulding

        % top level providing a first national classification
        % sub clustering of urban areas
            % selection criteria for to-be-subclustered classes

% generation of signature geometry
    % dissolution based on contiguity and assigned class


% - Feature importance - should this actually be included in here?